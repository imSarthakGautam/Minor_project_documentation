\chapter{LITERATURE REVIEW}


Youtube Recommendation System\cite{davidson2010youtube} delves into the intricacies of YouTube's video recommendation system, employing deep learning and a two-stage filtering approach to deliver personalized and pertinent content to users. It reviews three prevalent recommendation system methods—content-based filtering, collaborative filtering, and hybrid filtering—exploring their respective advantages and drawbacks. The YouTube model comprises two neural networks: a candidate generation network, selecting videos based on user activity, and a ranking network, scoring videos using numerous features and an objective function. The paper discusses research opportunities in addressing challenges faced by recommendation systems, including latency, cold-start, scalability, context-awareness, grey-sheep, and sparsity. It also outlines design principles for refining recommenders to mitigate these issues, offering insights into potential advancements and improvements in recommendation system architectures.
\\
A groundbreaking regularization approach\cite{article2} was presented for embedding layers in neural recommender systems, challenging the suitability of l2 regularization. Existing methods, including matrix factorization and graph convolutional networks, are reviewed, revealing limitations in approaches like data augmentation and joint learning. The proposed graph-based regularization leverages a kNN graph from user-item interactions, enforcing smoothness on embedding vectors based on Reproducing Kernel Hilbert Space and the graph Laplacian. Empirical evaluation on five datasets with two recommendation models demonstrates superior performance, especially for long-tail users and items, compared to l2 regularization. The article concludes by summarizing key contributions, including the novel regularization method and theoretical foundations, while suggesting future research directions such as exploring different graph types and incorporating side information. Overall, the study advances recommender system research by addressing the drawbacks of existing regularization methods and proposing an effective graph-based approach.
\\
Neural Network Collaborative Filtering with Amazon Book Reviews\cite{NN1} delves into creating a collaborating filtering Book Recommendation System using Amazon book reviews.The project uses a large dataset of Amazon book reviews from UCSD, and filters out books and users with less than 100 reviews. It makes use of Flair, a natural language processing toolkit, to obtain sentiment scores for each review, and transforms them using QuantileTransformers. It has applied a decomposition technique to create user and book embeddings from the user-book matrix, and uses weighted dot product to predict sentiment scores. After that a neural network model has been built using Tensorflow and Keras, with several dense, dropout, and batch normalization layers, to improve the prediction accuracy and account for complex interactions between embeddings.It shows an example of how the neural network model can recommend books to a user based on their preferences and review history.\\
Justifying Recommendations using Distantly-Labeled Reviews and
Fine-Grained Aspects \cite{NN2} describes the challenges of Justifying Recommendations using Distantly-Labeled Reviews and
Fine-Grained Aspects. The authors propose a pipeline to extract justifications from massive review corpora and annotate them with fine-grained aspects. They build large-scale personalized justification datasets from Yelp and Amazon Clothing. The authors develop two models for justification generation: (1)Ref2Seq, which leverages historical justifications as references and incorporates aspect-planning to guide generation; and (2) ACMLM, which is an aspect conditional masked language model that can generate diverse justifications from templates. The authors show that their models can generate convincing and diverse justifications that are relevant to users’ interests and preferences. They also provide an annotated dataset and a pipeline for justification extraction.\\
Using the Jaccard similarity method for recommendation system of books \cite{JS2} describes the application of the Jaccard similarity method to create appropriate reading lists for high school students based on their intellectual potential and age characteristics. It shows the use of the Jaccard similarity method, which is a mathematical formula that measures the similarity between two sets of words, to create a system of recommending books for high school students. It discusses the creation of a corpus based on literature textbooks of 5th-11th grade students and the comparison of this corpus with literary works in the Uzbek language. It shows the application of the Python programming language and the NLTK library to implement the Jaccard similarity method and to calculate the similarity scores between books. The results and analysis of the proposed method shows that it can recommend books with high similarity scores to the students’ potential and interests.\\
 A system that saves details of books purchased by
the user \cite{inproceedings} was proposed in 2016. From these Book contents and ratings, a hybrid algorithm using collaborative
filtering, content-based filtering and association rule generates book recommendations.
Rather than Apriori, they recommended the use of Equivalence class Clustering and
bottom up Lattice Transversal (ECLAT) as this algorithm is faster due to the fact that it examines the entire dataset only once.
\\
Research Problems in Recommender Systems\cite{article3} outlines the motivation to understand recommender systems and highlights research gaps. It also elucidates fundamental concepts like users, items, utility functions, and the recommendation process. It explores prevalent techniques in recommender systems, encompassing collaborative filtering, content-based, and hybrid methods, supplemented by real-world applications. It identifies research problems, including data-related challenges and changing user preferences, presenting existing solutions for each. The paper concludes by summarizing its main contributions. Key takeaways emphasize the ubiquity of recommender systems in diverse domains and the challenges they face, advocating for innovative solutions.
\\
In 2018, a model that generates recommendations to buyers\cite{okon} was proposed,
through an enhanced CF algorithm, a quick sort algorithm and Object Oriented Analysis
and Design Methodology (OOADM). Scalability was ensured through the implementation
of Firebase SQL. This system performed well on the evaluation metrics.
\\ 
Online Book Recommendation System using Collaborative Filtering \cite{JS} proposes a system that uses collaborative filtering with Jaccard similarity to recommend books to users based on their ratings and preferences. It explains how Jaccard similarity measures the ratio of common users who have rated two books to the total number of users who have rated either of the books, and how it can improve the accuracy of recommendations. It uses the Book Crossings Dataset with a collection of 2,71,379 books and 11,49,780 ratings from 2,78,858 users to test the performance of the proposed system, and compares it with existing algorithms. The paper reports the value of root mean square error (RMSE) as a metric to evaluate the system, and shows that the proposed system has a lower RMSE than other methods.\\
Matrix factorization recommendation algorithm \cite{article1} introduces  that employs a hybrid optimization approach, integrating Alternating Least Squares (ALS) and Stochastic Gradient Descent (SGD) in an advanced stage. The study compares the hybrid model with the individual ALS and SGD algorithms, evaluating their performance using the Mean Squared Error (MSE) metric. Notably, the hybrid optimization technique is designed to address the cold start problem, enhancing its real-world applicability. The paper systematically analyzes the strengths and weaknesses of each algorithm and the proposed hybrid model. By juxtaposing these techniques, the research aims to provide insights into the selection criteria for specific domains, offering a practical perspective on implementing recommendation systems. The utilization of MSE as a benchmark metric ensures a quantitative assessment of the effectiveness of the hybrid approach in comparison to ALS and SGD algorithms. 
\\
A paper with the framework to develop a
content based recommendation system  \cite{article4}for books was presented in 2018 which can further be integrated with a
collaborative filtering model. The proposed content based recommender will use the Named Entities as the basic criteria to rank books and give recommendations.
\\

