\chapter{CONCLUSION}

Thus, a recommendation system was realised using the best algorithm for recommending books by comparing among four commonly used algorithms.

To kick off the project, the Book Crossings dataset was collected from Kaggle. It contained the ratings and interactions on books from 90,000 users. This dataset was thoroughly analysed for any inconsistencies, outliers and null values. After preprocessing the dataset and formatting it into suitable forms, it was fed into the various models aforementioned in this report. A major obstacle during preprocessing was the abundance of ratings that were zero. Through study of the dataset, it was realized that the zero ratings were not actually rated zero by the user but they were not rated at all. This accounted for the sparsity in the dataset that was carefully handled during data preprocessing by filtering out the books that had been interacted with a certain number of times.

Additionally, the implementation details of each model also needs to be considered, as they significantly influence their performance and feasibility in real-world scenarios. The kNN model was implemented with a cosine similarity metric and varying values of k, requiring the storage of the entire dataset for efficient retrieval during recommendations. Matrix Factorization (MF) leveraged singular value decomposition (SVD) for dimensionality reduction, followed by collaborative filtering techniques to predict books. Jaccard Similarity (JS) utilized user-user similarity matrices to compute recommendations based on user-item interactions. Finally, the Neural Network (NN) model employed dense layers with appropriate activation functions and optimization techniques for training on the dataset. These implementation nuances, including parameter tuning and computational resources, played a crucial role in determining the models' effectiveness and scalability.

The key takeaway of this project is that the use cases of each recommendation models and instances when one should be preferred over others were indentified. Though the kNN model had overall best performance with the best Precision score, SVD was better prepared for the sparsity in the dataset shown by it's highly sensitive Recall. Jaccard similarity was able to better balance it's Precision and Recall as it has the less difference between Precision and Recall. Neural Network offers a balanced approach similar to Jaccard Similarity but with a higher Recall and lower Precision. However it's end F1-score was better compared to Jaccard Similarity.

All the objectives that were outlined when starting the project has been accomplished. For each of the recommendation models, their individual pros and cons were identified along with the trade offs of using one over another. A fully functional web-app 'Book Recommendation System' has been developed with the best performing model i.e KNN at the heart of it. The user interface of this web-app is simple to use and operate. At the end, the smooth operation of the components in the backend, showed that the system’s requirements, designs and implementation were handled correctly. All the deadlines have been met meaning that the schedule chosen was suitable for the project.
