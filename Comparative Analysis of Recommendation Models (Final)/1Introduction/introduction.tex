\chapter{INTRODUCTION}
% (20% of Proposal Length)
\pagenumbering{arabic}

\section{Background}

 Recommendations are a way to suggest people what they may like. Most of the recommendations were provided in person physically in past. People used to ask friends, nearby people for their suggestions. Then, with innovations and rise of AI, recommendation system were developed so as to not limit the suggestion to a selected group of people limited to a certain boundary. A recommendation system is a class of machine learning that uses data to help predict, narrow down, and find what people are looking for among an exponentially growing number of options. Likewise, books were bought with advice from friends and people nearby. However, it became a problem with availability of vast library of books and new titles releasing every minute. In order to simplify and solve the above problem, a book recommendation system is required. A simple user friendly website has been  developed to recommend books based on user preferences, genre, title, etc. This project provides precise recommendations to user's akin to their liking. This prevents the user from wasting a lot of valuable time on the internet.

\section{Motivation}

The motivation of our project is to discover the best way of recommending books to the user without having to deal with the problems that come with the recommendation process. Another motivation is to mitigate the overwhelming choices offered with release of so many books annually. This way users can save time and use that time to read more books and figure out their specific taste in books.We aim to recommend books based on item characteristics like: genre, author, etc. and their ratings rather than aggressively recommending with some fixed set of books. By using collaborative filtering method, the recommendation process will neither be too specific nor too general. Thus, helping users find books that remain in their orbit of preferences maybe even widening their horizons. Another motivation for our project is to let users enjoy the joy of discovery and be a part of it.

\section{Problem Statement}

Given the release of nearly 2 million new books in the market, it is a challenge to find a book that fits your tastes. The world has transitioned from a state of data scarcity on books to struggling surplus of data on the topic. It is hard to even keep tracks of sequels of books released that you liked. Similarly, there is a higher chance than ever to miss out on new releases of book that are just niche or lesser known. With the diversification of books from just plain writing stories, biography, etc. to more interactive format with pictures and drawings like: manga, kid's story book, etc., the choice has never been harder and options so limitless. With majority of recommendation system giving a very strong suggestions on books such that user might love the book or just hate the book completely. There is hardly a line drawn between being too specific or too general.

\section{Project Objectives}
The objective of our project is
\begin{itemize}
    \item To implement machine learning models for recommending books.
    \item To analyze the performance of machine learning models for book recommendation systems.
\end{itemize} 

\section{Scope of Project}

The project can be used by everyone for precise suggestions for books, novels, visual books, etc. The project uses collaborative algorithms so as to recommend books that strike a perfect balance between being too general and too specific. Users can analyze and check the book's metadata such as author's name, year of publication, genre and so on. Additionally, the system provides additional information about recommended books, reviews and ratings. Personalization will be a key focus of this project, ensuring that recommendations align with the user preferences. \\
The project however can only make recommendations based on the available books in datasets. Despite the dataset already having millions of books, in real-world scenarios the numbers are never sufficient. Since the project mostly utilizes collaborative filtering based on user interactions with books, The newly released books, or books that may be underrated by general users are harder to recommend. This is mainly due to the fact that newer books have novelty attached with them and will need to loosen the preprocessing constraints, which in turn might affect other general recommendation procedures negatively.

\newpage

\section{Potential Project Applications}

Some of the potential applications of our projects are:

\subsection{Libraries and Educational Institutes}
The ability to create user profile helps to suggest personalized books either for school work or novels and so on. Libraries can be aided to keep attracting more people as well as help students, customer and researchers to find relevant books aligned with their field of interest or academic course. Books can be catalogued based on genre, year of publication, author as well as cost and ratings.

\subsection{Bookstores}
A hybrid based recommendation system can recommend not just based on user behaviour but also on item characteristics which leads to more accurate recommendations. It can recommend top rated, trending books which can lead to higher sales for bookstores. User experience can be improved by providing tailored suggesting hence boosting sales. Bookstores can be suggested with books that align with current trends as main attraction to the store. 

\subsection{Ed tech companies}
Some students can learn things with the given course books while others find it extremely hard to grasp. Similar books but with easier learning reviews will be suggested to further help students with studies. The user profiles of students to further student's preferences and habits to increase the effectiveness of their courses can be used by Ed tech companies. This personalized learning aid can help improve the reputation the companies and attract more students in the near future.

\subsection{E-commerce and Online platform}
Most of the e-commerce sites in Nepal typically shows all listed books by arrival rather than recommending personalized suggestion. This system can be integrated on such platform to recommend books based on browsing history, demographic data as well as user behaviour to increase sales. Those sites can use the metadata of our system to show the ratings and impressions of the books to possible customers.

\section{Originality of Project}

\begin{itemize}
    \item The best performing model among some recommendation models has been identified.
    \item Insights on the strengths and weaknesses of each model have been gained.
    \item Based on the findings, the recommendation models has been optimized for specific application or use case.
\end{itemize}

\section{Organisation of Project Report}

The material in this project report is organised into 4 chapters. After this introductory chapter introduces the problem topic this research tries to address, chapter 2 contains the literature review of vital and relevant publications, pointing toward a notable research gap. Chapter 3 captures the requirements and the feasibility of the project. Chapter 4 describes the methodology along with it's architecture for the implementation of this project. Chapter 5 includes the implementation of algorithms in model along with the working process of the system. Chapter 6 provides an overview of the result and analysis of this project. Chapter 7 encapsulates Future Enhancement of this project. Chapter 8 highlights the conclusion of this project.  